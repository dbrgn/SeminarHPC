\begin{aufgabe}
Berechnen Sie die Julia-Mengen der komplexen Funktion $f_c(z)=z^2+c$ f"ur
beliebige Werte von $c\in\mathbb C$.
\end{aufgabe}

Das ber"uhmte ``Apfelm"annchen'' ist die Menge der Punkte $c\in\mathbb C$,
f"ur welche Iterationen
\[
f_c(0), f_c\circ f_c(0), \dots f_c^n(0),\dots
\]
der Abbildung $f_c$ beschr"ankt bleiben.
Punkte, f"ur die die Iterationen divergieren, werden auf den bekannten
graphischen Darstellungen jeweils mit verschiedenen Farben eingef"arbt,
die angeben, wie schnell die Divergenz erfolgt.

In dieser Aufgabe wird aber nicht der Parameterraum der Werte $c$
untersucht, sondern die Variable $z$.
Je nach Startwert $z_0$ konvergieren die Werte 
\[
z_1=f_c(z_0), z_2=f_c(z_1),\dots,z_{i+1}=f_c(z_i),\dots
\]
gegen einen Fixpunkt, oder gegen einen Zyklus von $p$ Werten (periodischer
Punkt mit Periode $p$), oder sie divergieren.
Die komplexe Ebenen l"asst sich je nach Verhalten in verschiedene
Gebiete einteilen. Die Julia-Menge ist die Grenze zwischen diesen
Gebieten.

Die Julia-Menge zeichnet sich dadurch aus, dass die Iteration von $f_c$ auf
der Julia-Menge chaotisch ist.
Dies ist aber f"ur die numerische Berechnung ein sehr schwierig auszuwertendes
Kriterium.
Beginnt man mit einem Punkt sehr nahe bei der Julia-Menge,
dann entfernen sich die Iterierten von der Julia-Menge,
da sie ja zu einem periodischen Punkt oder Fixpunkt hinstreben.
Man kann also die Julia-Menge dadurch aufsuchen, dass man die Urbilder sucht.
Bei einer quadratischen Funktion gibt es jeweils zwei Urbilder,
andere Funktionen k"onnen sehr viel mehr Urbilder haben. Die Urbilder
r"ucken immer n"aher an die Julia-Menge heran.



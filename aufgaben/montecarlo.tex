\begin{aufgabe}
Die Monte-Carlo Methode.
\end{aufgabe}

Probabilistische Algorithmen k"onnen deterministische Probleme l"osen,
insbesondere wenn keine direkten deterministischen Algorithmen bekannt 
sind.
Um eine vergleichbare Genauigkeit mit traditionellen Algorithmen zu
erreichen, ist eine sehr viel gr"ossere Zahl von Einzelresultaten notwendig.

Ein einfaches Beispiel f"ur einen Monte-Carlo Algorithmus ist die Integration
einer Funktion $f\colon[0,1]\to[1,0]$.
Statt das Integral direkt zu berechnen,
bestimmt man empirisch die Wahrscheinlichkeit, dass $P(Y < f(X))$, wobei
$X$ und $Y$ in $[0,1]$ gleichverteilte Zufallsvariablen sind. Dann gilt
\[
P(Y<f(X)) = \int_0^1f(x)\,dx.
\]
Eine Schwierigkeit bei der Durchf"uhrung einer Monte-Carlo-Simulation
ist, eine Quelle gen"ugend zuf"alliger Zufallszahlen zu bekommen.
Inbesondere hat man in Graphikkarten a priori keine guten
Zufallszahlgeneratoren. 

Ein alternatives Problem, welches man mit Monte-Carlo-Simulation untersuchen
k"onnte, handelt von der Akkumulation von Fehlern in einer Tr"agheitsplattform.
Eine solche muss zu Beginn auserichtet werden, was nat"urlich nicht
exakt m"oglich ist. Die Orientierung einer Tr"agheitsplattform wird gegeben
durch eine orthogonale Matrix in $\operatorname{SO}(3)$. 
Die tats"achliche Orientierung einer Tr"agheitsplattform ist also eine
Wahrscheinlichkeitsdichte auf $\operatorname{SO}(3)$, sie ordnet jeder m"oglichen
Orientierung eine gewisse Wahrscheinlichkeit zu.

Im Betrieb "andert sich die Orientierung der Plattform, durch Messung
der Winkelgeschwindigkeit kann man die wahrscheinlichste Orientierung
berechnen.
Unvermeidliche Messfehler der drei Winkelgeschwindigkeitssensoren
f"uhren aber dazu, dass die Ungenauigkeit immer gr"osser wird, die
Wahrscheinlichkeitsverteilung auf $\operatorname{SO}(3)$ wird immer ``verwaschener''.
Es gibt eine Vermutung, dass dies von einer Art Diffusions- oder
W"armeleitungsgleichung beschrieben wird.
Eine interessante Anwendung von Monte-Carlo-Simulation w"are, die
Entwicklung der Verteilung damit zu berechnen, und mit der 
Differentialgleichung zu vergleichen.

Historische Aspekte zur Monte-Carlo-Method:
\url{http://library.lanl.gov/cgi-bin/getfile?00326866.pdf}

\begin{aufgabe}
Visualisierung der Greenschen Funktion in zwei Dimensionen.
\end{aufgabe}

Gegeben ist eine quadratische, elektrisch leitende Platte
konstanten Widerstands, deren Rand geerdet ist.
Nun wird in einem Punkt $P_0=(x_0,y_0)$ nicht notwendigerweise
in der Mitte, eine Elektrode auf Potential $U_0$ angebracht.
Welches Potential $u(x,y)$ misst man im Punkt $(x,y)$?

Die Theorie sagt, dass man die L"osung dieses Problems bekommen
kann, indem man die partielle Differentialgleichung
\[
\frac{\partial^2u}{\partial x^2}
+
\frac{\partial^2u}{\partial y^2}
=f(x,y),
\]
auf dem Gebiet $\Omega=[0,1]^2=\{(x,y)\,|\, 0\le x,y\le 1\}$ 
mit der Randbedingung
$u(x,y)=0$ auf $\partial \Omega$
l"ost, wobei $f$ eine Funktion ist, die die Elektroden repr"asentiert.
F"ur diese spezielle Differentialgleichung gibt es ein L"osungsverfahren,
welches darauf beruht, die Gleichung f"ur Delta-Funktionen
$f(x,y)=\delta_P(x,y)$ an verschiedenen Punkten $P$ zu l"osen.

Ziel dieser Aufgabe ist eine Visualisierung der L"osung mit Hilfe der
Greenschen Funktion. Eine Visualisierung f"ur den eindimensionalen
Fall existiert bereits unter \url{http://www.youtube.com/watch?v=Wpi7Gf7V2HY}.
In dieser Simulation wird die rechte Seite $f$ der Differentialgleichung
durch eine Summe von Deltafunktionen approximiert, und zu jeder
N"aherungssumme die zugeh"orige L"osung berechnet. Diese Visualisierung
verwendete die Tatsache, dass sich die Greensche Funktion im eindimensionalen
Fall in geschlossener Form angeben l"asst.

Ziel dieser Aufgabe ist, eine analoge Visualisierung f"ur das 
oben beschriebene zweidimensionale Problem zu erstellen.
In diesem Fall l"asst sich die Greensche Funktion nur numerisch
bestimmen, bzw. man kann direkt die partielle Differentialgleichung
f"ur eine rechte Seite l"osen, die eine Summe von Delta-Funktionen ist.





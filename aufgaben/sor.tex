\begin{aufgabe}
SOR und SSOR: Konvergenzbeschleunigung f"ur iterative Solver f"ur
lineare Gleichungssysteme.
\end{aufgabe}

Im Skript werden der Jacobi- und der Gauss-Seidel-Algorithmus
behandelt. Mit Hilfe von Matrixzerlegungen wurde gezeigt, wie man
die Konvergenzgeschwindigkeit ermitteln kann. Successive OverRelaxation
(SOR) und Symmetric SOR (SSOR) sind Methoden, die Konvergenz zu
beschleunigen.

Ziel dieser Aufgabe ist an Beispielen und mit theoretischen 
"Uberlegungen zu zeigen, wie SOR die Konvergenz beschleunigt.

Literatur: David S.~Watkins, {\it Fundamentals of Matrix Computations},
Wiley, 2010 (in der HSR Bibliothek).

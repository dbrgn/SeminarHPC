\begin{aufgabe}
Als Motivationsbeispiel im Abschnitt "uber iterative Methoden f"ur lineare
Gleichungssysteme wurde die partielle Differentialgleichung mit
\[
\Delta u = f(x,y)
\]
Randbedingungen $u(x,y)=g(x,y)$ f"ur $(x,y)\in\partial\Omega$ dargestellt.
Man finde eine m"oglichst effiziente numerische L"osung f"ur dieses Problem.
\end{aufgabe}

Die Diskretisierung f"uhrt auf ein Gleichungssystem mit der Matrix
\[
A=\begin{pmatrix}
T&E& & &      & \\
E&T&E& &      & \\
 &E&T&E&      & \\
 & &E&T&      & \\
 & & & &\ddots& \\
 & & & &      &T
\end{pmatrix}
\]
Zerlegt man die Matrix als
\[
A
=
\underbrace{
\begin{pmatrix}
T& & & &      & \\
 &T& & &      & \\
 & &T& &      & \\
 & & &T&      & \\
 & & & &\ddots& \\
 & & & &      &T
\end{pmatrix}}_M
+
\underbrace{
\begin{pmatrix}
0&E& & &      & \\
E&0&E& &      & \\
 &E&0&E&      & \\
 & &E&0&      & \\
 & & & &\ddots& \\
 & & & &      &0
\end{pmatrix}}_N,
\]
erh"alt mein ein interatives Verfahren.
Die Matrix $M$ ist tridiagonal, diese spezielle Form einer Matrix ist
besonderes effizient invertierbar in Zeit $O(n)$.
Ausserdem ist
\[
M^{-1}=\begin{pmatrix}
T^{-1}&      &      \\
      &\ddots&      \\
      &      &T^{-1}
\end{pmatrix},
\]
Die Inverse von $M$ ist also sehr einfach aufgebaut,
insbesondere gibt es in der LAPACK-Library eigene Funktionen zur 
Invertierung solcher Matrizen.
Damit wird die Iterationsgleichung
$ x=M^{-1}(b - Nx)$
effizient implementierbar.

Vergleichen Sie die Performance dieses Verfahrens mit den Standard-Verfahren
von Jacobi und Gauss-Seidel.
Es ist auch denkbar, als Erweiterung den dreidimensionalen Fall zu untersuchen.

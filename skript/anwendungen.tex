\chapter*{Einf"uhrung}
Im zweiten Teil werden Anwendungsbeispiele aus der HPC-Welt gezeigt.
Eigentliche rechnerische L"osungen stehen hier neben mathematischen
"Uberlegungen, wie man ein Problem der L"osung mit leistungsf"ahiger
Rechnerhardware zug"anglich macht.

{\em Marco Bassotti} und {\em Dario Tr"utsch} verwenden OpenCL, um die R"uckprojektion
auf paralleler Hardware zu berechnen, sie erreichen dabei eine erstaunliche
Beschleunigung der Rechnung.
Die R"uckprojektion ist ein wesentlicher Schritt, um mit einem
Computertomographen aus den Rohdaten Bilder entstehen zu lassen.

{\em Nicol\'as Rom\'an L"uthold} und {\em Flavio La Morea}
wagen sich an das Problem,
eingen Kugelsternhaufen zu simulieren. Das Problem ist nicht einfach,
w"achst doch der Simulationsaufwand mit der dritten Potenz der Anzahl
der Sterne.
Es gelingt ihnen, trotz dieser widrigen Umst"ande, ein Video der Dynamik
eines kleinen Sternhaufens zu erstellen.

{\em Danilo Bargen} und {\em Lukas Murer} lassen die geballte Rechenleistung von GPUs
mit Hilfe von OpenCL auf das Problem los, Passwort-Hashes zu knacken.
Ihre Rechenbeispiele zagen, wie Passwort-Hashes von Passw"ortern mit nur
6 Stellen innert Sekunden geknackt werden k"onnen.

{\em Tabea M\'endez} und {\em Christian Schmid} f"uhren die Berechnung des aus der
Theorie der komplexen dynamischen Systeme bekannten ``Apfelm"annchens''
in Quaternionen durch. Da die Quaternionen einen vierdimensionalen
Raum bilden, w"achst die Datenflut bei Steigerung der Aufl"osung mit
der vierten Potenz an, und auch die Visualisierung dieser grossen Datenmengen
ist nicht trivial.


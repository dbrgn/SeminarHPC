\section{Implementation}
\rhead{Implementation}
\subsection{Wahl des Dateiformates}

Als Dateiformat w"ahlten wir FITS da diese Files auch als Bilder interpretiert werden k"onnen und auch \texttt{double}  Werte enthalten k"onnen. Der Punkt(0,0) befindet sich bei einem FITS File in der unteren linken Ecke. Dies ist wichtig im Bezug auf die richtige Darstellung, so dass die R"uckprojektion nicht gedreht und gespiegelt abgebildet wird. Bei den meisten anderen Dateiformaten befindet sich der Punkt(0,0) nicht in der unteren linken Ecke.

\subsection{Sequentielle Implementation}

Um einen Vergleich zu haben und das Prinzip besser zu verstehen
starteten wir mit einer sequentiellen Implementation. Der grossteil
der Implementation besteht aus einlesen und ausgeben von Files und der
Aufbereitung der Daten. Die eigentliche "Ubersetzung findet auf ein paar
wenigen Zeilen statt.

Um Rechenaufwand zu sparen berechneten wir vor der Transformation
die ben"otigten Sinus- und Cosinuswerte. Diese Werte speicherten
wir anschliessend in je einem Array und "ubergaben diese der
Transformationsfunktion.


\lstset{caption={Kern der sequentiellen Implementation}}
\begin{lstlisting}

	// Transformation	
	for(int i=0; i < anzPixel; ++i)
	{
		// Der Ursprung liegt im Zentrum des Outputbildes		
		int y = wide/2 - i%wide;  // Y-Wert des Outputbildes
		int x = i/wide - high/2;  // X-Wert des Outputbildes
		
		for(int angle=0; angle < high; ++angle)
		{
			// angle:  entspricht der Zeilennummer des Inputbildes			
			// radius: entspricht der horizontalen abweichung vom 
			// Zentrum des Inputbildes			
			radius = int(((x)*cosinus[angle] + (y)*sinus[angle])/1.41);
			output[i] += input[radius + wide/2 + angle*wide];
		}
	}
	// Skalierung aller Punkte um 1/PI	
	for(int i=0; i < anzPixel; ++i)
	{
		output[i] = output[i] * invPi;
 	}

\end{lstlisting}

\subsection{Parallele Implementation}

Bei der parallelisierten Implementation haben wir uns f"ur eine
OpenCL Implementation entschieden. Dies vor allem aus dem Grund, dass
sich eine OpenCL Implementation sehr gut eignet um viele von einander
unabh"angige Berechnungen auszuf"uren. Jeder Punkt in der R"uckprojektion
ist unabh"angig von allen andern, daher kann f"ur jeden Bildpunkt ein
eigenes Workitem erstellt werden. Die Adresse des Workitems eignet sich
auch bestens um die absolute Position des Bildpunktes im Outputbild
festzustellen.

Ein weiteres grosses Problem ist die Berechnung von Sinus-
und Cosinuswerten. Diese Funktionen sind schwer zu aproximieren,
ben"otigen viele Rechenoperationen und sind daher langsam. In unserem
Fall ben"otigen wir die Sinus- und Cosinuswerte in definierten und
konstanten Winkelabst"and. In diesem Fall kann man aus den Sinus- und
Cosinuswerten und den alten Werten die n"achsten Werte berechnen, dies
reduziert die Berechnung auf vier Multiplikationen und zwei Aditionen.

\begin{equation}
	\alpha = \text{const}
\end{equation}

Ohne Vereinfachung\\
\begin{equation}
	\text{Sinus}_n = \sin(n\cdot\alpha) 
\end{equation}
\begin{equation}
	\text{Cosinus}_n = \cos(n\cdot\alpha)
\end{equation}

Vereinfacht durch Iteration\\
\begin{equation}
	\text{Sinus}_{n+1} = \sin(\alpha)\cdot \text{Cosinus}_n + \cos(\alpha)\cdot \text{Sinus}_n
\end{equation}
\begin{equation}
	\text{Cosinus}_{n+1} = \cos(\alpha)\cdot \text{Cosinus}_n - \sin(\alpha)\cdot \text{Sinus}_n
\end{equation} \\


Bei einer OpenCL-Implementation f"uhrt jedes Workitem einen sogenannten
Kernel aus. Ein Kernel ist eine relativ kurze Codesequenz, welche einer
Funktion "ahnelt. Sie enth"alt aber zus"atzliche Schl"usselw"ortern f"ur
OpenCL. Der Kernel wird in C geschrieben und in einem separaten File
gespeichert. Im main() wird OpenCL initialisiert, der Kernel eingebunden,
die Hardware aquiriert und so weiter. Bei unserer Implementation ist
die Initialisierung, gemessen an der Codel"ange, um Faktoren gr"osser
als der Kernel.

\lstset{caption={Kern des Kernels der parallelen Implementation}}
\begin{lstlisting}

	// Transformation
	
	// Index fuer dieses Workiten im Outputbild errechnen
	__private int	index = get_global_id(0) + get_global_id(1) * get_global_size(0);	
	// Sinus und Cosinus fuer winkel 0
	__private double cosinus = 1;
	__private double sinus = 0;
	// x/y Koordinaten im Outputbild
	__private int y = get_global_size(0)/2 - get_global_id(0);
	__private int x = get_global_id(1) - get_global_size(1)/2;
	// Laufwariable	
	__private int	angleNr = 0;
	// Index des Imputstrings
	__private int	pixIndex = 0;
	// Berechnung
	while(angleNr < high)
	{
		// Koordinaten des Inputbildes		
		radius = ((x)*cosinus + (y)*sinus)/1.4142;
		pixIndex = (radius + offset) + angleNr*wide;
		// Summation der Werte der gefundenen Punkte
		temp += input[pixIndex];
		// Berechnung der naechsten 
		// Sinus und Cosinus Werte
		sinus = sinA*cosinus + cosA*sinus;
		cosinus = cosA*cosinus - sinA*sinus;
		angleNr++;
	}
	// Skalierung
	temp = temp * invPi;
	// Rueckgabe eines Bildpunktes
	output[index] = temp;

\end{lstlisting}


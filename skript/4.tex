Tridiagonale Gleichungssysteme sind effizient l"osbar mit einer Variante des
Gauss-Algorithmus, die auch als Thomas-Algorithmus bekannt ist.
F"ur die Matrix 
\[
A=\begin{pmatrix}
2.1&1&0&1\\
1&2.1&1&0\\
0&1&2.1&1\\
1&0&1&2.1
\end{pmatrix}
\]
ist jetzt auch eine Zerlegung in eine tridiagonale Matrix
\[
M=\begin{pmatrix}
2.1&1&0&0\\
1&2.1&1&0\\
0&1&2.1&1\\
0&0&1&2.1
\end{pmatrix}
\]
und $N=A-M$ m"oglich.
Lohnt sich der zus"atzliche Aufwand des Thomas-Algorithmus, oder
anders gefragt:
Welches Iterationsverfahren konvergiert schneller,
Gauss-Seidel oder das zu dieser Zerlegung geh"orige Verfahren?

\begin{loesung}
F"ur das Gauss-Seidel-Verfahren muss man den Spektralradius von
\[
C_{\text{GS}}=
\begin{pmatrix}
2.1&0&0&0\\
1&2.1&0&0\\
0&1&2.1&0\\
1&0&1&2.1\\
\end{pmatrix}^{-1}
\begin{pmatrix}
0&1&0&1\\
0&0&1&0\\
0&0&0&1\\
0&0&0&0
\end{pmatrix}
\]
berechnen, man findet $\varrho(C_{\text{GS}})=0.90756$.
Andererseits findet man f"ur 
\[
C_{\text{T}}=
\begin{pmatrix}
2.1&1&0&0\\
1&2.1&1&0\\
0&1&2.1&1\\
0&0&1&2.1\\
\end{pmatrix}^{-1}
\begin{pmatrix}
0&0&0&1\\
0&0&0&0\\
0&0&0&0\\
1&0&0&0
\end{pmatrix}
\]
den Spektralradius $\varrho(C_{\text{T}})=0.83969$,
der zus"atzliche Aufwand bei der L"osung
des tridiagonalen Gleichungssystems schl"angt sich tats"achlich in
rascherer Konvergenz nieder.
\end{loesung}

\chapter{Julia-Mengen}
\rhead{Julia-Mengen}
\begin{refsection}

\chapterauthor{Andreas M"uller}
\section{Was sind Julia-Mengen?}
F"ur jede komplexe Zahl $c\in\mathbb C$ kann man die Iterationen der
Funktion
\begin{equation}
f_c(z)=z^2 + c
\label{julia:quadratic}
\end{equation}
untersuchen.
Ausgehend von einem Anfangswert $z_0\in\mathbb C$ bilden wir die Folge
\[
z_1=f(z_0),\;z_2=f(z_1),\;z_3=f(z_2)\;\dots, z_{n+1}=f(z_n),\;\dots
\]
F"ur $|z|\ge |c|$ und $|z|\ge 3$ folgt
\[
|f_c(z)|= |z^2+c|\ge |z|^2-|c|\ge |z|^2-|z|=|z|(|z|-1)\ge |z|\ge 2|z|.
\]
Die Iterierten von $z_n$ wachsen exponentiell schnell "uber alle Grenzen.

Es ist aber auch m"oglich, dass f"ur einzelne Punkte die Folge $z_n$ 
beschr"ankt bleiben. Im besten Fall werden sie gegen einen Punkt
$\hat z$ konvergieren.
Der Punkt $\hat z$ ist ein Fixpunkt von $f_c$, denn
\[
\hat z = \lim_{n\to \infty}f_c^{n}(z)\quad\Rightarrow\quad
f_c(\hat z)=f_c(\lim_{n\to\infty}f_c^n(z))=\lim_{n\to\infty}f_c^{n+1}(z)=\hat z.
\]
Die Folge $z_n$ muss aber nicht unbedingt gegen einen Punkt konvergieren,
es kann auch einen Zyklus von $k$ Punkten $\hat z_1,\dots,\hat z_k$ geben
so, dass
\[
z_{kn+1}\to \hat z_1,\quad
z_{kn+2}\to \hat z_2,\quad
z_{kn+3}\to \hat z_3,\quad\dots,\quad
z_{kn+k}\to \hat z_k.
\]

Die komplexe Ebene besteht also aus h"ochstens zwei Regionen: dem Bereich
von komplexen Zahlen, die bei Iteration "uber alle Grenzen wachsen, und
einem Bereich von Punkten, die bei Iteration gegen einen Zyklus konvergieren.
Dazwischen gibt es eine Menge von Punkten, f"ur den die Iterationen weder
zum einen noch zum anderen Verhalten f"uhren.
Auf dieser Menge ist die Iteration chaotisch, sie
heisst die Julia-Menge von $f_c$.

\begin{beispiel}
Im Fall $c=0$ ist $f_0(z)=z^2$. Da jede komplexe Zahl geschrieben werden
kann als $z=re^{i\varphi}$, sind die Iterierten von $f_0$:
\[
z_1=r^2e^{2i\varphi},
z_2=r^4e^{4i\varphi},
z_3=r^8e^{8i\varphi},
z_4=r^{16}e^{16i\varphi},\dots,
z_n=r^{2^n}e^{2^ni\varphi},\dots
\]
Wenn $r < 1$ ist, dann werden die Iterierten $z_n$ immer kleiner, die
Folge konvergiert gegen $0$.
Falls $r>1$ dann wachsen die Iterierten "uber alle Grenzen.
Im Falle $r=1$ bleiben die Iterierten auf dem Einheitskreis.
Die komplexe Ebene wird also in drei Bereiche geteilt: das
Innere des Einheitskreises besteht aus Zahlen, deren Iterierte gegen
0 konvergieren, das "Aussere des Einheitskreises besteht aus Zahlen,
deren Iterierte gegen $\infty$ konvergieren.
Die Grenze zwischen den beiden Bereichen ist der Einheitskreis selber,
die Iterierten der Zahlen auf dem Einheitskreis konvergieren im
allgemeinen nicht.

Der Einheitskreis ist die Julia-Menge der Abbildung $f_c$ f"ur $c=0$.
\end{beispiel}

F"ur Werte $c\ne 0$ des Parameters kann die Julia-Menge sehr kompliziert
werden, so kompliziert, dass sie nur dank Computer-Berechnungen
ausreichend detailliert visualisiert werden kann.

Die Iteration der quadratischen Abbildungen (\ref{julia:quadratic})
wurden sorgf"altig studiert, sie sind relativ einfach, zeigen aber
trotzdem viele Eigenschaften der nichtlinearen dynamischen Systeme.
Insbesondere zeigt schon $f_0$ auf dem Einheitskreis chaotisches
Verhalten \cite{julia:devaney}.

Die Julia-Mengen werden derart kompliziert, dass man ihnen keine ganzzahlige
Dimension mehr zuweisen kann. Vergr"ossert man einen Ausschnitt, werden
immer mehr Details sichtbar, die ``Kurve'' wird immer komplizierter.
Ist ist oft m"oglich, den Julia-Mengen eine gebrochenzahlen Dimension
zuzuweisen.
Die Definition dieser gebrochenzahligen Dimensionen ist nicht ganz
selbstverst"andlich, Mengen mit gebrochenzahliger Dimension treten
aber mindestens n"aherungsweise in der Natur immer wieder auf, sie
sind auch als Fraktale bekannt \cite{julia:falconer}.

In den Abbildungen~\ref{julia:a} bis \ref{julia:h} sind Julia-Mengen
f"ur eine Auswahl von Parameterwerten $c$ visualisiert.
F"ur \ref{julia:a} bis \ref{julia:f} gibt es jeweils einen Zyklus, gegen
den ein Teil der Iterationsfolgen konvergiert.
In \ref{julia:g} und \ref{julia:h} ist die Menge der Startpunkte, die zu einem
Zyklus f"uhren auf die leere Menge zusammengeschrumpft, es bleibt nur
noch die Julia-Menge "ubrig, bestehend aus Startwerten, die nicht zu
Iterationen gegen $\infty$ f"uhren.
In \ref{julia:g} zerf"allt die Julia-Menge in einzelne Punkte, 
man spricht auch von ``Cantor-Staub'',
in \ref{julia:h} besteht sie aus einem fraktalen Baum.

\section{Berechnung von Julia-Mengen}
Es gibt zwei M"oglichkeiten, eine Julia-Menge sichtbar zu machen.
In den F"allen, wo ein Fixpunkt $\hat z$ existiert, kann man f"ur jeden
Punkt der komplexen Ebene die Iterationsfolge $z_n$ berechnen, und
entscheiden ob sie gegen $\infty$ oder einen Zyklus konvergiert.
Die Julia-Menge ist dann die Grenze zwischen diesen Bereichen.

Dies funktioniert allerdings nicht f"ur Parameterwerte, f"ur die
es keinen Zyklus gibt, gegen den die Folge $z_n$ konvergieren k"onnte.
In diesem Fall kann man die Abbildung $f_c(z)$ umkehren: wenn die
Punkte $z_n$ von der Julia-Menge weg gegen $\infty$ konvergieren, dann 
m"ussen die Urbilder gegen die Juliamenge konvergieren.
Man kann also die Julia-Menge dadurch finden, dass man alle Urbilder
bestimmt.

\subsection{Julia-Mengen durch Iteration}
Um ein Bild der Julia-Menge zum Parameter $c$ zu erhalten, berechnet
man f"ur jeden Startpunkt $z_0\in\mathbb C$ die Folge $z_n$ und
bestimmt den kleinsten Wert $n$, f"ur den $|z_n|>b$ ist, wobei $b$
irgend eine fest Schranke ist.
Der Pixel zum Punkt $z_0$ wird mit einer Farbe eingef"arbt, die von $n$
abh"angt.
Falls $z_n$ die Grenze $b$ nicht "uberschreitet, bleibt der Pixel schwarz.

Die Berechnung der Iterationsfolge ist f"ur jeden Pixel vollkommen
unabh"angig von jedem anderen Pixel, dieses Beispiel ist als pr"adestiniert
f"ur die Berechnung mit OpenCL.
Jeder Pixel stellt ein eigenes Work-Item dar.
Das Programm {\tt julia1.c} im Repository implementiert diese Strategie.

Diese Methode zeigt nicht die Julia-Menge, sondern die Menge der Startpunkte,
die unter Iteration nach $\infty$ konvergieren. Den Bereichen von Startpunkten,
die zu beschr"ankten Iterationsfolgen f"uhren, kann mit dieser Methode
keine Farbe zugeordnet werden.
In den Abbildungen~\ref{julia:a} bis \ref{julia:h} sind diese Bereiche 
schwarz dargestellt.

\subsection{Julia-Mengen durch Inverse Iteration}
Die hier beschriebene Methode zur Berechnung von Julia-Mengen wird in
\cite{julia:peitgenrichter} als Inverse Iteration Method bezeichnet.
F"ur die meisten Startpunkte konvergiert die Iterationen der Abbildung
$z\mapsto f_c(z)$ konvergiert gegen $\infty$
oder einen Zyklus, insbesondere entfernen sich die Punkte $z_n$ von der
Julia-Menge.
Verfolgt man die Iteration r"uckw"arts, also indem man eine Folge
$\tilde z_{n+1}=f_c^{-1}(\tilde z_n)$ konstruiert, dann n"ahern sich die
Punkte der Julia-Menge. Indem man die Punkte $\tilde z_n$ f"ur grosse $n$
plottet, erh"alt man eine Apprxomation der Julia-Menge.

Die Konstruktion der Folge $\tilde z_n$ ist nicht eindeugt, denn die
Gleichung
\[
z_n= f_c(z_{n+1})=\tilde z_{n+1}^2+c
\]
hat zwei L"osungen $\pm\sqrt{z_n-c}$. 
Um eine m"oglichst gute Approximation der Julia-Menge zu bekommen, 
werden wir jede m"oglich Wurzel r"uckw"arts verfolgen wollen.
Die zwei Urbilder von $z$ haben unter $f_c$ insgesamt vier Urbilder, 
die wiederum acht Urbilder.
Jede weitere Iteration verdoppelt die Anzahl der Urbilder, daf"ur ist
in OpenCL kein Platz vorhanden, wir k"onnen nun mal in OpenCL nich beliebig
Speicher allozieren.

Wir verwenden daher eine andere Strategie: wir w"ahlen jeweils nur eines
der Urbilder, wobei wir Zufallszahlen verwenden, um eines der zwei
Urbilder zu w"ahlen.
Diese Berechnung wiederholen wir viele Male, jeweils mit einer anderen
Zufallszahlfolge.
Auf diese Weise werden ebenfalls alle Punkte der Julia-Menge irgendwann
approximiert.

Mit dieser Strategie handeln wir uns aber ein neues Problem ein, wir brauchen
eine Folge von Zufallszahlen.
OpenCL stellt kein API f"ur Zufallszahlen zur Verf"ugung, dies muss also
ebenfalls im OpenCL Code implementiert werden.
Die Anforderungen an die Qualit"at der Zufallszahlen ist nicht besonders
hoch, ein einfacher Zufallszahlgenerator auf der Basis von linearen
Kongruenzen wie der Lehmer-Generator \cite{julia:lehmer},
auch bekannt als Park-Miller Generator, reicht v"ollig aus.

Die Inverse Iteration f"uhrt f"ur stark verzweigte Julia-Mengen nicht
zu guten Resultaten.
Man kann dies dank der fraktalen Natur der Julia-Menge verstehen.
Beginnt man mit einer geschlossenen Kurve $\gamma_0$ in $\mathbb C$,
dann sind die Urbilder $\gamma_{n+1} = f_c(\gamma_n)$ jeweils Vereinigungen
von geschlossenen Kurven, die die Julia-Menge immer besser Approximieren.
Wegen der fraktalen Natur der Julia-Menge wird $\gamma_{n}$ mit zunehmendem
$n$ immer l"anger.
Entsprechend wird die Dichte der durch Inverse Iteration
erzeugten Punkte auf $\gamma_{n+1}$ immer kleiner. 
F"ur Julia-Mengen, die sich wie Kurven verhalten, ist die Approximation
recht gut (Abbildungen~\ref{julia:a} und \ref{julia:h}).
Am anderen Ende des Spektrum befinden sich Julia-Mengen, die in Cantor-Staub
zerfallen wie in Abbildung~\ref{julia:g}, welche nur sehr schlechte
Approximationen haben.

\begin{figure}
\begin{center}
\includegraphics[width=\hsize]{julia/a.png}

\bigskip

\includegraphics[width=\hsize]{julia/j-a.png}
\end{center}
\caption{Julia-Menge f"ur $c= -0.1+0.1i$\label{julia:a}}
\end{figure}

\begin{figure}
\begin{center}
\includegraphics[width=\hsize]{julia/b.png}

\bigskip

\includegraphics[width=\hsize]{julia/j-b.png}
\end{center}
\caption{Julia-Menge f"ur $c= -0.5+0.5i$\label{julia:b}}
\end{figure}

\begin{figure}
\begin{center}
\includegraphics[width=\hsize]{julia/c.png}

\bigskip

\includegraphics[width=\hsize]{julia/j-c.png}
\end{center}
\caption{Julia-Menge f"ur $c= -1+0.05i$\label{julia:c}}
\end{figure}

\begin{figure}
\begin{center}
\includegraphics[width=\hsize]{julia/d.png}

\bigskip

\includegraphics[width=\hsize]{julia/j-d.png}
\end{center}
\caption{Julia-Menge f"ur $c= -0.1+0.75i$\label{julia:d}}
\end{figure}

\begin{figure}
\begin{center}
\includegraphics[width=\hsize]{julia/e.png}

\bigskip

\includegraphics[width=\hsize]{julia/j-e.png}
\end{center}
\caption{Julia-Menge f"ur $c= 0.25+0.52i$\label{julia:e}}
\end{figure}

\begin{figure}
\begin{center}
\includegraphics[width=\hsize]{julia/f.png}

\bigskip

\includegraphics[width=\hsize]{julia/j-f.png}
\end{center}
\caption{Julia-Menge f"ur $c= -0.5+0.55i$\label{julia:f}}
\end{figure}

\begin{figure}
\begin{center}
\includegraphics[width=\hsize]{julia/g.png}

\bigskip

\includegraphics[width=\hsize]{julia/j-g.png}
\end{center}
\caption{Julia-Menge f"ur $c= 0.66i$\label{julia:g}}
\end{figure}

\begin{figure}
\begin{center}
\includegraphics[width=\hsize]{julia/h.png}

\bigskip

\includegraphics[width=\hsize]{julia/j-h.png}
\end{center}
\caption{Julia-Menge f"ur $c= -i$\label{julia:h}}
\end{figure}

\printbibliography[heading=subbibliography]
\end{refsection}

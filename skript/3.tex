Sei $A_a$ die Matrix
\[
A_a=\begin{pmatrix}a&1\\1&a\end{pmatrix}.
\]
F"ur welche Werte von $a$ ist das Gauss-Seidel-Verfahren f"ur diese Matrix
konvergent?

\begin{loesung}
Das Gauss-Seidel-Verfahren entspricht der Zerlegung
\[
M_a=\begin{pmatrix}a&0\\1&a\end{pmatrix},\qquad
N=\begin{pmatrix}0&1\\0&0\end{pmatrix}.
\]
Die Matrix $M^{1}$ kann mit Hilfe von Minoren berechnet werden:
\[
M^{-1}=\frac1{a^2}\begin{pmatrix}
a&0\\
-1&a
\end{pmatrix}
=
\begin{pmatrix}
\frac1{a^2}&0\\
-\frac1a&\frac1{a^2}
\end{pmatrix}
\]
Die Matrix
\[
C=M^{-1}N=\begin{pmatrix}
0&\frac1{a^2}\\
0&-\frac1a
\end{pmatrix}
\]
hat das charakteristische Polynom 
\[
\chi_C(\lambda)=\lambda\biggl(\lambda+\frac1{a^2}\biggr)
\]
mit den Nullstellen $\lambda=0$ und $\lambda=-\frac1{a^2}$. Insbesondere gilt
\[
\varrho(M^{-1}N)=\frac1{a^2}=\begin{cases}
>1\qquad &|a| < 1,\\
=1\qquad &|a|=1,\\
<1\qquad &|a| > 1.
\end{cases}
\]
Das Verfahren ist also genau f"ur $|a|>1$ konvergent.
\end{loesung}


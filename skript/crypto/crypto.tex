\chapter{Knacken von MD5 mit OpenCL}
\rhead{Knacken von MD5 mit OpenCL}
\begin{refsection}

\chapterauthor{Danilo Bargen, Lukas Murer}

\section{Einleitung}

- Was ist ein Hash-Algorithmus?
- Was ist MD5?
- Was ist Brute Force Cracking?
- Überlegungen zum Keyspace
- Zielsetzung

\section{Probleme mit grossen Problemen}

- Wo stossen wir an unsere Grenzen?
- Was gibt es für L"osungsans"atze?

\section{L"osungsansatz}

Wie haben wir es umgesetzt, und warum?

- Grob-Ansatz, Bezug auf vorherige Section
- Herleitung Wort

\section{Implementationseigenheiten}

(Ev. als Subsection von "L"osungsansatz")
- Limitationen der gew"ahlten L"osung
- Was k"onnte man verbessern?
- ...

\section{Resultate}

- Performancemessungen
- Bedeutung f"ur Umgang mit Passw"ortern

\printbibliography[heading=subbibliography]
\end{refsection}

F"uhren Sie f"ur das Gleichungssystem
\[
\begin{linsys}{3}
x&+&y& & &=&1\\
x&+&y&+&z&=&2\\
 & &y&+&z&=&3
\end{linsys}
\]
je einen Abstiesschritt ausgehend vom Punkt $x_0=(1,0,-1)$ durch, und zwar
\begin{teilaufgaben}
\item
mit Abstiegsrichtung
\[
\vec v=\begin{pmatrix}1\\0\\0\end{pmatrix}
\]
\item
mit steilstem Abstieg
\end{teilaufgaben}

\begin{loesung}
F"ur den Abstieg verwenden wir die Abstiegsformel
\[
x_1=x_0-\frac{p^t(Ax_0-b)}{p^tAp}p
\]
\begin{teilaufgaben}
\item
Wegen
\[
Ax_0=
\begin{pmatrix}
1&1&0\\
1&1&1\\
0&1&1
\end{pmatrix}
\begin{pmatrix}
1\\0\\-1
\end{pmatrix}
=
\begin{pmatrix}1\\0\\-1\end{pmatrix}
\qquad
\text{und}
\qquad
Ax_0-b
=
x_0-b
=
\begin{pmatrix}
0\\-2\\-4
\end{pmatrix}
\]
($x_0$ ist ein Eigenvektor von $A$) folgt
$v^t(Ax_0-b)=0$.
Der Abstiegsschritt f"uhrt also auf den gleichen Punkt: $x_1=x_0$.
\item
Steilster Abstieg verlangt, dass als Abstiegsrichtung die Richtung des
Gradienten von $J$ verwendet wird. 
Dieser ist
\[
p=\operatorname{grad}J(x_0)=Ax_0-b=\begin{pmatrix}0\\-2\\-4\end{pmatrix}.
\]
Die Abstiegsformel f"ur diese Richtung liefert
\begin{align*}
p^tp&=20,
&
p^tAp&=\begin{pmatrix}0\\-2\\-4\end{pmatrix}^t
\begin{pmatrix} -2\\-6\\-6 \end{pmatrix}
=12+24=36,
&
x_1&=x_0-\frac{20}{36}p
=\begin{pmatrix}
1\\
\frac{10}{9}\\
\frac{11}{9}
\end{pmatrix}.
\end{align*}
\end{teilaufgaben}
\end{loesung}

\begin{diskussion}
Die Matrix $A$ in der letzten Aufgabe ist nicht positiv definit,
man kann also nicht erwarten, dass das Verfahren konvergiert.
\end{diskussion}

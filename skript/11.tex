Mit Hilfe der Potenzmethode kann man auch Nullstellen eines beliebigen
Polynoms finden. Wenden Sie diese Methode auf das Polynom
\[
p(x)=x^3-6x^2+11x-6
\]
an, und verwenden Sie den Startvektor
\[
x_0=\begin{pmatrix}1\\1\\1\end{pmatrix}.
\]
Finden Sie die betragsgr"osste Nullstelle von $p(x)$.

\begin{loesung}
Die zu $p$ geh"orige Matrix ist
\[
A_p
=
\begin{pmatrix}
6&-11&6\\
1&  0&0\\
0&  1&0
\end{pmatrix}
\]
Die Iteration mit $x_0$ ergibt:
\[
Ax_0=
\begin{pmatrix}
6&-11&6\\
1&  0&0\\
0&  1&0
\end{pmatrix}
\begin{pmatrix}1\\1\\1\end{pmatrix}
=
\begin{pmatrix}
1\\1\\1
\end{pmatrix},
\]
also ist $1$ ein Eigenwert von $A_p$, oder $p$ hat $1$ als Nullstellen.
Allerdings ist
\[
q(x)=\frac{p(x)}{x-1}=\frac{x^3-6x^2+11x-6}{x-1}
=x^2-5x+6,
\]
und $q(x)$ hat die beiden Nullstellen
\[
x_\pm=\frac52\pm\sqrt{\frac{25}4-6}
=\frac52\pm\sqrt{\frac14}=\frac52\pm\frac12=\begin{cases}3\\2.\end{cases}
\]
Der Betragsgr"osste Eigenwert von $A_p$ ist also 3.

Der Grund, warum die Potenzmethode den Betragsgr"ossten Eigenwert nicht 
findet ist, dass wir die Iteration mit einem Eigenvektor von $A_p$
beginnen.
Da Eigenvektoren von $A_p$ als Elemente Vielfache der Potenzen des
zugeh"origen Eigenwertes haben, sollte man die Iteration mit einem
Vektor beginnen, dessen Komponenten nicht monoton zunehmend oder
abnehmend sind. Der Startvektor
\[
x_0=\begin{pmatrix}1\\\frac12\\\frac14\end{pmatrix}
\]
f"uhrt auf die Iteration
\begin{align*}
A_px_0
&=
\begin{pmatrix}
6&-11&6\\
1&  0&0\\
0&  1&0
\end{pmatrix}
\begin{pmatrix}1\\\frac12\\\frac14\end{pmatrix}
=
\begin{pmatrix}
6-\frac{11}2+\frac64
1\\
\frac12
\end{pmatrix}
=
\begin{pmatrix}
2\\
1\\
\frac12
\end{pmatrix}
=2
\begin{pmatrix}1\\\frac12\\\frac14\end{pmatrix}
\end{align*}
Da die Komponenten dieses Startvektors in Bin"ararithmetik exakt
dargestellt werden k"onnen, wird man mit diesem Startvektor ebenfalls
den betragsgr"ossten Eigenwert nicht finden.

In Octave ist es zweckm"assig, einen zuf"alligen Startvektor zu w"ahlen:
\begin{center}
\begin{tabular}{|>{$}c<{$}|>{$}r<{$}>{$}r<{$}>{$}r<{$}|}
\hline
k& x_{k,1}&x_{k,2}&x_{k,3}\\
\hline
 0&  0.03254&   0.53063&  0.27617\\
 1&  1.00000&  -0.00816& -0.13316\\
 2&  1.00000&   0.18900& -0.00154\\
 3&  1.00000&   0.25564&  0.04831\\
 4&  1.00000&   0.28753&  0.07350\\
 5&  1.00000&   0.30505&  0.08771\\
% 6&  1.00000&   0.31538&  0.09620\\
% 7&  1.00000&   0.32175&  0.10148\\
% 8&  1.00000&   0.32577&  0.10482\\
% 9&  1.00000&   0.32837&  0.10697\\
10&  1.00000&   0.33005&  0.10838\\
%11&  1.00000&   0.33116&  0.10930\\
%12&  1.00000&   0.33189&  0.10991\\
%13&  1.00000&   0.33237&  0.11031\\
%14&  1.00000&   0.33270&  0.11058\\
15&  1.00000&   0.33291&  0.11076\\
%16&  1.00000&   0.33305&  0.11088\\
%17&  1.00000&   0.33314&  0.11095\\
%18&  1.00000&   0.33321&  0.11101\\
%19&  1.00000&   0.33325&  0.11104\\
20&  1.00000&   0.33328&  0.11106\\
%21&  1.00000&   0.33330&  0.11108\\
%22&  1.00000&   0.33331&  0.11109\\
%23&  1.00000&   0.33332&  0.11110\\
%24&  1.00000&   0.33332&  0.11110\\
25&  1.00000&   0.33333&  0.11110\\
26&  1.00000&   0.33333&  0.11111\\
\hline
\end{tabular}
\end{center}
Mit diesem Startvektor konvergiert die Iteration gegen einen Eigenvektor
zum Eigenwert 3, dem betragsgr"ossten Eigenwert.
\end{loesung}


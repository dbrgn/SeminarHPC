F"ur welche Werte von $a$ und $b$ ist das Gauss-Seidel-Verfahren f"ur die
Matrix
\[
A=\begin{pmatrix}
a&1\\
1&b
\end{pmatrix}
\]
konvergent?

\begin{loesung}
Dem Gauss-Seidel-Verfahren entspricht die Zerlegung
\[
M=\begin{pmatrix}a&0\\1&b\end{pmatrix},\qquad
N=\begin{pmatrix}0&1\\0&0\end{pmatrix}.
\]
Die Matrix
\[
C=M^{-1}N
=
\frac1{ab}
\begin{pmatrix}b&0\\-1&a\end{pmatrix}
\begin{pmatrix}0&1\\0&0\end{pmatrix}
=
\begin{pmatrix}
0&\frac1a\\0&-\frac1{ab}
\end{pmatrix}
\]
hat das charakteristische Polynom
\[
\chi_C(\lambda)=\lambda\biggl(\lambda +\frac1{ab}\biggr)
\]
mit den Nullstellen
\[
\lambda=\begin{cases}
0&\\
\displaystyle
-\frac1{\sqrt{ab}},&
\end{cases}
\]
der Spektralradius ist also
\[
\varrho(M^{-1}N)=\frac1{\sqrt{|ab|}}.
\]
Das Gauss-Seidel-Verfahren ist also genau dann konvergent, wenn
\[
|ab| > 1.
\]
\end{loesung}

